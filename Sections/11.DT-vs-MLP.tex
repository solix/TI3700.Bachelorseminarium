In one paper a Decision Support System to detect oil spills has been created. First, SAR images go through a segmentation process to highlight the dark spots. Second, features are calculated for each dark spot that serve as input for the classification method.  For classification both a neural network and a decision tree are used. Since both classifiers are designed and tested using the same data, a comparison on accuracy can be made. The neural network appears to outperform the decision tree slightly. In addition, parallelization techniques are discussed which is important for decision support system to operate in real time using high resolution SAR images.\cite{Mera201472}
	
Another attempt has been made to compare different classifiers for oil spill detection.\cite{Xu201414} Here they experienced an opposite result. Out of seven classifiers, the MLP was found to perform the worst by 14.8 percentage points. The results indicated that the additional flexibility provided by SVM and MLP does not necessarily improve the predicitive performance compared to less flexible methods. Both tend to overfit the dataset on small-sized training samples and could not generalize well.
	
Both classifiers were also used for the automatic diagnosis of Hashimoto's disease using ultrasound thyroid images\cite{Omiotek201340}. This case is similar to the detection of oil spills where preprocessing of the image, feature extraction and classification is performed. 50 feaures were used as input to a MLP and decision tree. The same classification accuracy for sick patients has been achieved using the two different classifiers.

Multiple studies\cite{Topouzelis200762}\cite{Delfrate200038}\cite{Topouzelis200930}\cite{Topouzelis200924}Oil spills\cite{Delfrate2004} have been done on classifying oil spills using only MLP. Their performance range from $76$\% to $96$\% in classification accuracy which can be found in the table. MLP seems to be a popular choice due to it's ability to handle nonlinear data, but decision trees employ a set of rules for classification which is easier to understand and interpret.






	