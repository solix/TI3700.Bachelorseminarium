In one paper a Decision Support System to detect oil spills has been created. For classification both a neural network and a decision tree are used. Since both classifiers are designed and tested using the same data, a comparison on accuracy can be made. The neural network appears to outperform the decision tree slightly. \cite{Mera201472}
	
	Another attempt has been made to compare different classifiers for oil spill detection.\cite{Xu201414} Here they experienced an opposite result. Out of seven classifiers, the ANN was found to perform the worst by 14.8 percentage points. The results indicated that the additional flexibility provided by SVM and ANN does not necessarily improve the predicitive performance compared to less flexible methods. Both tend to overfit the dataset on small-sized training samples and could not generalize well.
	
	Both classifiers were also used for the automatic diagnosis of Hashimoto's disease using ultrasound thyroid images\cite{Omiotek201340}. This case is similar to the detection of oil spills where preprocessing of the image, feature extraction and classification is performed. 50 feaures were used as input to a ANN and decision tree. The same classification accuracy for sick patients has been achieved using the two different classifiers.


	