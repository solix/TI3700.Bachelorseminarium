Oil spill detection can be done with several methods. In the field of optical imaging alone, there are plenty of options. Aircrafts can create images across various spectra of light with different cameras. After heating, oil emits certain levels of infra red radiation. Another method is to measure the reflectance, since oil reflects light differently than water. Laser fluorsensors measure the emission fluorescent light when oil is hit by ultraviolet light. But all these methods have several downsides as well. Though reflectance is higher in oil, it does not show a distinct pattern in reflecting light \cite{fingas2014review}. This makes it hard to distinguish between oil and other reflective surfaces. Also, this technique can only be applied when there is light. Infra red techniques requires heat to function.
SAR has a high resolution, wide area coverage and moreover images can be taken day and night under any weather condition. This enables oil spill investigators to monitor oceans 24 hours a day\cite{Chang20081915}. The advantage of this is that large areas can be checked for spills, which is often required when searching through entire oceans. 
The main alternative within radar imaging is Side-Looking Airborne Radar(SLAR), which is an older but significantly cheaper method \cite{fingas2014review}. 

SAR images are created through radio waves that are emitted from the SAR device to an area. That area reflects the radio waves. The SAR device measures the time the waves take to travel back  \cite{Doerry:04}. When an ocean is hit by radio waves, those waves are reflected in a certain way. Oil reflects radio waves differently. This difference is the key to identifying oil spills.
In the image that is drawn from these measurements, water is shown as a grey or lighter area, while a possible oil spill is shown as a dark spot.