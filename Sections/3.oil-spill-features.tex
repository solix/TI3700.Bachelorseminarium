In order to classify oil candidates as either oil spills or look-alikes, features from dark spots are extracted and then compared to predefined values. Around 25 features are commonly used in the scientific community \cite{Topouzelis200930}. These can be divided into three major catagories\cite{Brekke200595}:
\begin{enumerate}
\item geometric characteristics (e.g. area, perimeter, complexity)
\item Physical characteristics of the backscatter level of the spot and its surroundings (e.g. mean, max backscatter value)
\item Contextual information (e.g. ships present, distance to shore)
\item Texture (e.g. mean contrast)
\end{enumerate}
Most classifiers rely heavily on geometric shape features and the contextual feature.\cite{Xu201414} Researchers have tried to reduce the amount of features to counter the curse of dimensionality and the risk of overfitting. Features belonging to the same catagory appear to be highly correlated\cite{Xu201414}, leading to the search for a subset of features with less redundancy and retains most of the predicitive power\cite{Topouzelis200930}. 
