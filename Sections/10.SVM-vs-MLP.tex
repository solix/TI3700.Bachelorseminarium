According to Wolpert and Macready \cite{wolpert1995no}, `any two optimization algorithms are equivalent when their performance is averaged across all possible problems'. This makes .  

Researchers \cite{Moavenian20103088,Zanaty2012177} have run some experiments to know which classifier might predict more accurate than others; the outcome of these experiments indicated that SVM outperforms MLP in one case while in another type of dataset SVM scored lower than MLP. This supports Wolpert and Macreadys' statement. 

MLPs are popular in machine learning. They excel in fields like speech recognition, image recognition because of their ability to solve problems stochastically, meaning that it allows user to get approximate solution for very complex problems like fitness approximation\cite{jin2005neural}. If we look closely at SVM and MLP performance using real world dataset in various disciplines(ECG arrhythmias,text recognition, remote sensing, wind speed prediction) we see that there is no free lunch \cite{wolpert1995no}

In the field of ECG(Electrocardiography), comparison between MLP and SVM is done using same dataset. MLP accuracy in classifying the ECG(Electrocardiography) signals, is more accurate than other ANN. MLP with back propagation(BP) training algorithm suffer from slow convergence to local minima, on the other hand SVM classifier with (K-A) training do not trap in local minima points therefore they are faster than ANN\cite{Moavenian20103088}.\\
 In high dimensional data, SVM accomplishes better accuracy compare to MLP, this is when new kernel function proposed for SVM. The reason behind is that MLP needs more hidden units for tested data set and become more complex when the dimension of data set increases whereas SVM complexity does not depend on dimension of data set. SVM are efficient on optimal separation of unseen data points\cite{Zanaty2012177}.\\
SVM works better than MLP for the off-line signature recognition(within finite database). The comparison is done between the identification rate(increment of 20\% for SVM) and for training time needed. The superiority of SVM is because of its generalization ability in high dimensional space\cite{FriasMartinez2006693}.\\
SVM outperforms to MLP, in Wind speed prediction. The comparison is done using a wind speed dataset that cover a 12 years between 1970-1982. Dataset is divided into three parts: training, test and validation sets. The output result shown the SVM outperform MLP on all orders resulting in the lower MSE(MLP is 0.0090 while it is 0.0078)\cite{Mohandes2004939}.\\
In oil spill detection, the datasets are imbalanced, in which most instances(that are not oil spill spot) belong to a larger class and very few instances for smaller class(oil spill spot). Lack of instances in minority class will cause SVM suffer from biased decision boundary and prediction performance, however combining SVM with an over-sampling and under-sampling technique will make SVM outperform to other classifiers\cite{liu2006boosting}.

	
      




	