%In this section, We will investigate comparisons between three classifiers SVM, DT and MLP, in different areas of study, to measure their performance and accuracy in their prediction.\\
%In order to have an indicator on suitable classifier in certain discipline, user needs to tune some concrete parameter such as dimensionality of feature, size of dataset and dataset balance to be able to  decide on performance ratio, mean squared error(MSE) , accuracy or recognition rate of output; for example SVMs and MLP outperform to DT, when dealing with high dimensional feature whereas for classification problems that deals with discrete features, DT is preferred.\cite{kotsiantis2007supervised}.  
 
According to the free lunch theorem \cite{wolpert1995no}, not one of two learning algorithms can outperform the other avaraged on all possible problems. Luckily, we only focus on oil spill detection where differences between classifiers can exist. In this section we will investigate the three classification techniques pairwise to highlight their main differences. According to Kotsiantis(2007), the simplest approach in deciding which algorithm is most appropiate for a particular classification problem is to estimate their accuracies and select the one that seems most accurate\cite{kotsiantis2007supervised}. An overview of almost all forthcoming research results is given in \ref{fig:table}. A task is considered similar to oil spill detection when it is part of remote sensing or applies a similar methodology as in \ref{fig:overview}, preferably using a low dimensional feature space.