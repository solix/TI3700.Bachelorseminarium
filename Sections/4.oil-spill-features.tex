In order to classify oil candidates as either oil spills or lookalikes, features of dark spots are extracted and then compared to predefined values. Around 25 features are commonly used in the scientific community and can be found in a table \cite{Topouzelis200930}. These can be divided into three major categories\cite{Brekke200595}:
\begin{enumerate}
\item geometric characteristics (e.g. area, perimeter, complexity)
\item Physical characteristics of the backscatter level of the spot and its surroundings (e.g. mean, max backscatter value)
\item Contextual information (e.g. ships present, distance to shore)
\item Texture (e.g. mean contrast)
\end{enumerate}

Geometric characteristics are the features that are related to the shape of the dark spot and is applied by all methods in the table\cite{Topouzelis200930}. An important feature is elongatedness which can be expressed as a ratio between the width and length of the dark spot\cite{Gasull20071}. Features that use the backscatter level of the spot and its surroundings take into account the gradient level in the image. For example, the background standard deviation is a feature that is highly effected by wind level and is generally high for lookalikes. Contextual features incorporate other information mostly not extractable from the image. These include the distance between a dark spot and a ship, weather information and whether the dark spot lies in a frequently polluted area. Texture provides information about the spatial correlation among neighbouring pixels.

Most classifiers rely heavily on geometric shape features and the contextual feature.\cite{Xu201414} Researchers have tried to reduce the amount of features to counter the curse of dimensionality and the risk of over-fitting. Features belonging to the same category appear to be highly correlated\cite{Xu201414}, leading to the search for a subset of features with less redundancy and retains most of the predictive power\cite{Topouzelis200930}. The values of these features are extracted and used as inputs in a chosen classifier.
