Marine oil Spill is a form of human made environmental pollution usually occur during transportation of oil, drilling platforms or tankers \cite{Zhang201476}or during the maintenance on oil exploration sites in the ocean. One of the largest occurrences of oil spills in the history happened in the Gulf of Mexico as the oil spread out in the ocean by the explosion on the drilling platform affected marine ecosystem and wild life\cite{Bozeman2011244}.\\
Detection of oil spills can help to control environmental risks and prevent incalculable damages by it. Synthetic aperture radar(SAR) image is useful to detect oil spills spots in the ocean. \\
There are various automated systems proposed in order to detect spills spots in SAR images. These systems analyze the SAR images, assigns the probability of dark spots and proposes an algorithm to classify the dark images in to the oil spills and look alike shapes \cite{Xu201414,brekke2008classifiers,Keramitsoglou2006640,Guo2014146}.\\
We will start this article by looking at different system for detecting oil spills, but we will focus on SAR. Afterwards, we will examine the preprocessing of SAR images. Thirdly, an explanation of the features used in oil spill detection is given. We shortly touch on how they are extracted and why these are commonly used. We then quickly explain the three different supervised learning classification methods we want to compare. In section four we compare our three classifiers with each other. We then have a discussion based on the results of section four. Here we look at which classifier is a good choice for oil spill detection. We finalize our paper with a conclusion.