Marine oil spill is a form of human made environmental pollution that usually occurs during transportation of oil, extracting oil with drilling platforms \cite{Zhang201476} or during the maintenance on oil exploration sites in the ocean. One of the largest occurrences of oil spills in history happened in the Gulf of Mexico, the oil spread out in the ocean caused by the explosion on the drilling platform affected marine ecosystem and wild life\cite{Bozeman2011244}. The New York Times wrote about the aftermath \cite{bpnytimes}. In their article they discuss the massive financial consequences for the oil company BP, as they ended up paying 27 billion dollars to clean up the oil.

Detection of oil spills can help to control environmental risks and prevent damages. Synthetic aperture radar(SAR) image is useful to detect oil spills spots in the ocean. \\
There are various automated systems in order to detect spills spots in SAR images. These systems analyze the SAR images, identify possible oil spills and use an algorithm to classify oil spills and lookalike shapes \cite{Xu201414,brekke2008classifiers,Keramitsoglou2006640,Guo2014146}.

We will start this article by looking at different systems for detecting oil spills, but we will focus on SAR. Afterwards, we will examine the preprocessing of SAR images. Thirdly, an explanation of the features used in oil spill detection is given. We shortly touch on how they are extracted and why these are commonly used. We then quickly explain the three different supervised learning classification methods we want to compare. In section four we compare our three classifiers with each other. Next, we'll have a discussion based on the results of section four. Here we look at which classifier is a good choice for oil spill detection. We finalize our paper with a conclusion and recommendations.