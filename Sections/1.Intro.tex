Marine oil Spill is a form of human made environmental pollution usually occur during transportation of oil, drilling platforms or tankers \cite{Zhang201476}or during the maintenance on oil exploration sites in the ocean. One of the largest occurrences of oil spills in the history happened in the Gulf of Mexico as the oil spread out in the ocean by the explosion on the drilling platform affected marine ecosystem and wild life\cite{Bozeman2011244}.\\
Detection of oil spills can help to control environmental risks and prevent incalculable damages by it. Synthetic aperture radar(SAR) image is useful to detect oil spills spots in the ocean. SAR image has a high resolution, wide area coverage and moreover images can be taken day and night under any weather condition. This enables oil spill investigators to monitor oceans 24 hours a day\cite{Chang20081915}.\\
There are various automated systems proposed in order to detect spills spots in SAR images. These systems analyzes the SAR images, assigns the probability of dark spots and proposes an algorithm to classify the dark images in to the oil spills and look alike shapes \cite{Xu201414,brekke2008classifiers,Keramitsoglou2006640,Guo2014146}.\\
In machine learning, classification of oil spills and lookalikes in SAR images is of the highest importance. Main purpose of this article is to  investigate three popular supervised learning classifiers in general. Later on, a qualitative comparison between these classifiers in order to derive the criteria in choosing algorithms and classifiers that would be most suitable for detection of oil spills in SAR images. 