This paper presented a comparison between three popular classification techniques for the detection of oil spills using SAR images. Few papers employed decision trees or SVM which is why studies of similar problems that are related to oil spill detection have been included. Not all papers agreed on a particular classifier having the highest accuracy, but this could have been expected since the results are under discussion and direct comparison is difficult. Each experiment uses a different data set, some even included unverified samples. The chosen classifier largely depends on the complexity of the problem, feature dimension, type of features and the data set. Even for seemingly similar problems this can lead to the usage of a different classifier. SVM and MLP are recommended when the dataset is complex, because of their ability to handle non-linear classes. Single decision trees can be considered when training time and ease of interpretation matter, when data is scarce or when there is a lot of noise in the image. 

In the case of oil spill detection, much improvement in the research about different classifiers can be done by establishing a common database with labelled dark spots that are verified. Preferably, dark spots in all shapes, sizes and on different geographical points. This will allow researches to train a classifier with the same data set. Further research could include more data and a combination of classifiers.

To further increase the availability of data, we suggest more research is done on bagging. Sampling datasets multiple times can at least increase the amount of data available for training or validation purposes. Perhaps it can even help balance some datasets.

Another interesting option to increase data availability, is the fusion of images. This is done to fuse different types of data together, namely LandSat and SAR. Perhaps a similar technique can be used to extend SAR images. Again, this would not allow for new real data, but might help with training classifiers.

Finally, few studies use used Random Forests. These seem to give interesting results when used for classification of oil spills. They have some of the advantages of decision trees but can also handle more complex data. On the other hand, they might act more like a black box, losing the advantage of intuitiveness. This would be an interesting classifier to research.