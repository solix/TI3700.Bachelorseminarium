This paper presented a comparison between three popular classification techniques for the detection of oil spills using SAR images. Few papers employed decision trees or SVM which is why studies of similar problems that are related to oil spill detection have been included. Not all papers agreed on a particular classifier having the highest accuracy, but this could have been expected since the results are under discussion and can't really be compared. Each experiment uses a different data set, some even included unverified samples. The chosen classifier largely depends on the complexity of the problem, feature dimension, type of features and the data set. Even for seemingly similar problems this can lead to the usage of a different classifier. SVM and MLP are recommended when the dataset is complex, because of their ability to handle non-linear separable classes, though DT forests are also an option. Single decision trees can be considered when training time and ease of interpretation matter, when data is scarce or when there is a lot of noise in the image. 

In the case of oil spill detection, much improvement in the research about different classifiers can be done by establishing a common database with labelled dark spots that are verified. Preferably, dark spots in all shapes, sizes and on different geographical points. This will allow researches to train a classifier with the same data set. Further research could include more data and a combination of classifiers.

%Maybe it is possible to fuse SAR images together to create new training data for oil spill classifiers.