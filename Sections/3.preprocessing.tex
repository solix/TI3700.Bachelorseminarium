There are many factors that can cause problems in SAR images. The biggest problem is the presence of lookalikes. Distinguishing between oil spills and lookalikes is the toughest part of the classification process. Grease ice, rain cells and passing vessels can all be mistaken for oil spills\cite{Brekke200595}. In some cases, even a human expert cannot distinguish between a spill and a lookalike \cite{Keramitsoglou2006640}. Wind and organisms in the water are also factors to take into account when reading these images. To have better classification of oil spills within SAR images, preprocessing is done. \\
First, there is a general quality assessment. If the image is blurred in general, it will be impossible to get any information out of it. Afterwards, experts look at noise removal. One of the most common versions of noise is speckle noise. Speckle noise is caused naturally by a disruption in the phase of radio waves. Radio waves leave the sensor all in the same phase. Once they interact with an object, they scatter and are out of phase with each other. If they interact with each other, the result will be a darker or lighter pixel then is to be expected. Multi-look processing and spatial filtering can be done to reduce speckle noise \cite{simard1998analysis}. Experts also flag areas in the image which might interfere with the classification process. These areas include shorelines and land and ships, as these would all be represented by dark areas in the images.
High or low wind areas are also taken into consideration as they change the surface of the water and oil slicks. Another possibility is that dark spots are caused by algae or seaweed infestations\cite{fingas2014review}.  By minimizing the number of dark spots in the image, classification becomes easier and more accurate.
