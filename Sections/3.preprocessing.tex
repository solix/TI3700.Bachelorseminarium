But there are many factors that cause problems. The biggest problem is the presence of lookalikes. Distinguishing between oil spills and lookalikes is the toughest part of the classification process. Grease ice, rain cells and passing vessels can all be mistaken for oil spills\cite{Brekke200595}. In some cases, even a human expert cannot distinguish between a spill and a lookalike \cite{Keramitsoglou2006640}. Wind and organisms in the water are also factors to take into account when reading these images. To have better classification of oil spills within SAR images, preprocessing is done. First, there is a general quality assessment. Afterwards, experts look at speckle removal and noise removal. Experts also flag areas in the image which might interfere with the classification process. These areas include shorelines and land, high or low wind areas and but also algae infestations and seaweed dense areas\cite{fingas2014review}.
