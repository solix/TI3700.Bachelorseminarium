A decision tree is a classifier that makes several sequential decisions. The outcome of that sequence determines whether a data point belongs to one class or another. The structure of a decision tree \ref{fig:DT} is defined as a set of nodes ${x_1, ... , x_n}$ with edges ${e_1, .., e_{m}}$ between them. The edges are directed and there are no cycles in the network. The tree has one root $x_1$. With each data point we start at the root. The root node is fed information about a certain feature. Each node contains a rule that allows the node to make a decision. Each decision leads to a different node, where another decision is made. This process repeats itself until one of the leaf nodes is reached \cite{safavian1991survey}. From there we can determine what class our initial data point belonged to (either class $A$, $B$ or $C$ in the figure below. 
\begin{figure}[H]
    \includegraphics[width=80mm]{./img/decisiontree.png}
    \caption{Decision Tree example. $x_i$ are nodes where decisions are made. $e_j$ are edges leading down the tree to other nodes. $A,B,C$ are classes that instances are to be classified into.}
    \label{fig:DT}
\end{figure}

In general, the more nodes a tree has, the more accurate it's classification will become. The downside is that the time to classify and train will increase the larger the tree becomes. It is impossible to optimize the accuracy and efficiency at the same time. The design of the tree is crucial, as each node splits up the dataset in a certain way. Another problem is that errors may add up from node to node. The advantage of the decision tree is it's computational efficiency, even with multivariate analyses \cite{safavian1991survey}.


Decision trees are generally built with a simple algorithm. Find a feature that splits up the dataset into different classes the best way. That feature will be a node. For each of the subsets that comes forth out of that division, the process is repeated until there is enough certainty that the remaining subset represents the class. This can be determined through several methods. \\

(In the first draft there will be more information on how leaf nodes are determined, how dataset can be divided into subsets, different types of decision trees)