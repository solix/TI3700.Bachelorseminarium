Oil spills cause big problems, financially and ecologically. To prevent problems and react accordingly to oil spills, detection is required. But the detection process is not easy, as oil spills can be hard to identify. This is why different classifiers are used to support the decisions of experts. In this paper we look specifically at the classification of oil spills in Synthetic Aperture Radar(SAR) images. We compare three commonly used classifiers: Support Vector Machines, Decision Trees and MultiLayer Perceptrons. We compare their accuracy and overall characteristics, specifically when handling oil spills and SAR images. In the end we give recommendation based on the comparison of our classifiers and problems we encountered during the research.