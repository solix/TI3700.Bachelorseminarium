Marine oil spills cause big problems, financially and ecologically. To prevent and react accordingly to oil spills, detection is required. This detection process is not easy and successful discrimination of oil spills and lookalikes(e.g. algae, grease ice, rain cells and low wind areas) is dependent on the classification techniques used. This is why different classifiers are studied that support the decisions of experts in an automatic detection system. In this paper we look specifically at the classification of oil spills in Synthetic Aperture Radar (SAR) images. We compare three commonly used classifiers: Support Vector Machines (SVM), Decision Trees (DT) and MultiLayer Perceptrons (MLP) based on their accuracy and overall characteristics in order to give a recommendation. A wide range of papers using the different classifiers is discussed and their performance compared. It is hard and inappropriate to directly compare classifiers in terms of accuracy since they all use different datasets and different features. MLP is often used in oil spill detection and increasingly in the more broad area of remote sensing. MLP and SVM are well suited when large number of labelled samples are available because of their ability to handle non-linear data, but they are prone to over fitting. The more easy to understand DT should be considered when data is scarce. We recommend that the oil spill community creates a shared database. Also, further research on bagging and image fusion to increase the availability of data in this field should be done. Random Forests and SVM require more attention of the oil spill detection community, as they seem to be interesting candidates for the classification of oil spills.
