\hspace{0.5cm} Oil spill detection can be done with several methods. There are different types of data that can be used. In the field of images even, there are plenty of options. Aircrafts can create images across various spectra of light with different cameras. However, in this paper we focus on Synthetic Aperture Radar. SAR creates radar images with good spatial resolution and at a great distance. The main alternative is Side-Looking Airborne Radar, which is older but significantly cheaper. (M Fingas, C. Brown (2014), Review of oil spill remote sensing, Spill Science, Edmonton, Alberta,Canada)

These images are created through radio waves. Radio waves are sent from the SAR device to an area. That area reflects radio waves in a certain way, which is measured through the delay in times the wave takes to travel back. When an ocean is hit by radio waves, that energy is reflected in a certain way. Oil reflect that energy in a different way. A three-dimensional array of scene elements is defined which will represent the volume of space within which targets exist. Each element of the array is a cubical voxel representing the probability (a "density") of a reflective surface being at that location in space. (Wikipedia, will find other sources). This voxel density represents a difference in material (oil or water).

But there are many factors that cause problems. Wind, weather, fresh water and organisms in the water are all factors to take into account when reading these images. To have better classification of oil spills within these images, preprocessing is done. First the general there is a general quality assessment. Afterwards they look at speckle removal, noise removal. They also filter out areas in the image which might interfere with the detection process. These areas include shorelines, high or low wind areas and but also algea infestations and seaweed dense areas. These areas are flagged before the classification is done. (M Fingas, C. Brown (2014), Review of oil spill remote sensing, Spill Science, Edmonton, Alberta,Canada)

