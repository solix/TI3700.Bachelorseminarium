\hspace{0.5cm} Oil spill detection can be done with several methods. There are different types of data that can be used. In the field of images alone, there are plenty of options. Aircrafts can create images across various spectra of light with different cameras. However, in this paper we focus on Synthetic Aperture Radar. SAR creates radar images with good spatial resolution and at a great distance. The main alternative within radar imaging is Side-Looking Airborne Radar, which is an older but significantly cheaper method \cite{fingas2014review}. 

SAR images are created through radio waves. Radio waves are sent from the SAR device to an area. That area reflects radio waves in a certain way, which is measured through the time the wave takes to travel back. When an ocean is hit by radio waves, those waves is reflected in a certain way. Oil reflects waves in a different way. A three-dimensional array of scene elements is defined which will represent the volume of space within which targets exist. Each element of the array is a cubical voxel representing the probability (a "density") of a reflective surface being at that location in space \cite{wikireplacethis}. This voxel density represents a difference in material (oil or water).

But there are many factors that cause problems. Wind, weather, fresh water and organisms in the water are all factors to take into account when reading these images. To have better classification of oil spills within SAR images, preprocessing is done. First, there is a general quality assessment. Afterwards, experts look at speckle removal and noise removal\cite{Keramitsoglou2004}. Experts also flag areas in the image which might interfere with the classification process. These areas include shorelines and land, high or low wind areas and but also algae infestations and seaweed dense areas\cite{fingas2014review}.



