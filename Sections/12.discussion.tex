Here is an overview of all the results of previously mentioned studies.

\begin{table*}[t]
\centering
\setstretch{0.5}
\tabcolsep=0.09cm
\footnotesize
\begin{tabular}{*{6}{c}}
  Study & data type & Preprocessing & \# Features & formations & Results \\
    \hline
  
    Hydro-acoustics & Sonar & Echoview & 15 &  - & SVM accuracy: $89.5$\%, DT accuracy: $86.8$\% \\

    Land coverage1986 & LandSat SAR & "Using data miner" & 11 &  - & SVM max accuracy: $90.53$\%, DT accuracy: $93.48$\% \\

    Land coverage2001 & LandSat SAR & "Using data miner" & 10 &  - & SVM max accuracy: $93.67$\%, DT accuracy: $94.07$\% \\ 
    
    Oil spills\cite{Topouzelis200762} & ERS-2 SAR, 24 high-res images & 8-bit, transformation, Filtering, data normalization & 10 & 90 lookalikes and 69 oil spills & MLP(10:51:2) accuracy: $86.67$\% lookalike acc. $91.18$\% oil spills acc.\\
    
    Oil spills\cite{Delfrate200038} & ERS SAR, 600 low-res images & a) Resampling,Radiometric range correction, georeference & 11 & 68 lookalikes and 71 oil spills & MLP(11-8-4-1) accuracy: $90$\% lookalike acc. $82$\% oil spills acc. [leave-one-out approach]\\
    
    Oil spills\cite{Topouzelis200930} &  ERS-2 SAR, 24 high-res images & - & 10 & 90 lookalikes and 69 oil spills & MLP(10-51-2) accuracy: $84.4$\% lookalike acc. $85.3$\% oil spills acc.\\
    
    Oil spills\cite{Topouzelis200924} &  ERS SAR, 12 high-res images & 8-bit transformation, filtering & - & - & MLP(4-2-1) accuracy: $96.46$\% overall acc. \\
 
    Oil spills\cite{Delfrate2004} &  ERS SAR, 70 images & - & 12 & 78 lookalikes and 111 oil spills & MLP(12-8-8-1) 0.227 root mean square error(rmse)\\
    
    Oil spills\cite{Xu201414} &  RADARSAT-1, 93 images & log-transformation, standardization & 15 & 94 lookalikes and 98 oil spills & MLP $75.93$\% overall acc. SVM $79.63$\% overall acc. Decision tree(Bundling) 90.74 overall acc.\\
    
    Oil spills\cite{Mera201472} &  Envisat, 47 images & - & 9 & 155 lookalikes and 80 oil spills & MLP(9:11:2) $96.3$\% lookalike acc. $92.9$\% oil spill acc. Decision tree $92.6$\% lookalike acc. $92.9$\% oil spill acc.\\
    
    Oil spills\cite{Delfrate1996} &  ERS-1 SAR, 59 low-res images & - & 9 & 2471 lookalikes and 42 oil spills & Decision tree $96$\% lookalike acc. $86$\% oil spill acc\\
    
    Oil spills\cite{Topouzelis201268} &  ERS-2 SAR, 24 high-res images & - & 9 & 90 lookalikes and 69 oil spills & Decision tree forest $84.4$\%\\ 
    
    Oil spills\cite{brekke2008classifiers} & ENVISAT, 103 images & masking & - & 12244 lookalikes and 41 oil spills & SVM(C-SVC) $77.4$\% lookalike acc. $82.9$\% oil spill acc.\\
    
    
    
\end{tabular}
\end{table*}


In section 4, an attempt to compare the three classifiers has been made including similar experiments that are comparable to oil spill detection. These papers used a similar amount of features that where extracted in scarcely available images. \\ Not all papers agreed on a particular classifier having the highest accuracy, but this could have been expected since the results are under discussion and can't really be compared. Each experiment uses a different data set, some even included unverified samples. The chosen classifier largely depends on the feature dimension, type of features and the data set. Even for seemingly similar problems this can lead to the usage of a different classifier. In the case of oil spill detection, much improvement in the research about different classifiers can be done by establishing a common database with labelled dark spots that are verified. Preferably, dark spots in all shapes, sizes and on different geographical points. This will allow researches to train a classifier with the same data set.