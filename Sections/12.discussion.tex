There are some key issues for effectively classifying dark spots, several factors should be taken into account\cite{Kubat:1998:MLD:288808.288812}. The available data containing oil spills is scarce compared to lookalikes. This leads to a very imbalanced dataset. As a consequence, a classifier sensitive to this problem can not reach it's maximum accuracy\cite{Japkowicz20026}. SVM seem the least affected by this imbalance. Validity of the data selection, there is no guarantee that the data used for training the classifier are representatives of future samples, espescially with such a scarce dataset as in oil spill detection. The feature selection procedure should be done with care as it can influence prediction accuracy. Out of the 25 common features, most researchers arbitraly select a subset of features and compare multiple subsets using cross-validation, a technique for assessing how well a model generalizes to an independent dataset. Finally, a classifier should be chosen with all the issues mention in mind including the interpretablity and training times in a highly dynamic environment. In the table\ref{fig:table}, all previous mentioned studies are shown including their results and characteristics. This should allow a better discussion on which classifier is well suited for oil spill detection.

\begin{table*}[t]

\advance\leftskip-3cm
\setstretch{1.5}
\tabcolsep=0.19cm
\footnotesize

\adjustbox{max width=0.97\pdfpagewidth, left}{

\begin{tabular}{*{6}{c}}
  Study & data type & Preprocessing & \# Features & formations & Results \\
    \hline
  
    Hydro-acoustics \cite{Robotham2011170} & Sonar & Echoview & 15 &  - & SVM accuracy: $89.5$\%, DT accuracy: $86.8$\% \\

    Land coverage1986 \cite{Otukei2010S27} & LandSat SAR & "Using data miner" & 11 &  - & SVM max accuracy: $90.53$\%, DT accuracy: $93.48$\% \\

    Land coverage2001 \cite{Otukei2010S27} & LandSat SAR & "Using data miner" & 10 &  - & SVM max accuracy: $93.67$\%, DT accuracy: $94.07$\% \\ 
    
    Oil spills\cite{Topouzelis200762} & ERS-2 SAR, 24 high-res images 8-bit & transformation, Filtering, data normalization & 10 & 90 lookalikes and 69 oil spills & MLP(10:51:2) accuracy: $86.67$\% lookalike acc. $91.18$\% oil spills acc.\\
    
    Oil spills\cite{Delfrate200038} & ERS SAR, 600 low-res images & Resampling,Radiometric range correction, georeference & 11 & 68 lookalikes and 71 oil spills & MLP(11-8-4-1) accuracy: $90$\% lookalike acc. $82$\% oil spills acc.\\
    
    Oil spills\cite{Topouzelis200930} &  ERS-2 SAR, 24 high-res images & - & 10 & 90 lookalikes and 69 oil spills & MLP(10-51-2) accuracy: $84.4$\% lookalike acc. $85.3$\% oil spills acc.\\
    
    Oil spills\cite{Topouzelis200924} &  ERS SAR, 12 high-res images & 8-bit transformation, filtering & - & - & MLP(4-2-1) accuracy: $96.46$\% overall acc. \\
 
    Oil spills\cite{Delfrate2004} &  ERS SAR, 70 images & - & 12 & 78 lookalikes and 111 oil spills & MLP(12-8-8-1) 0.227 root mean square error(rmse)\\
    
    Oil spills\cite{Xu201414} &  RADARSAT-1, 93 images & log-transformation, standardization & 15 & 94 lookalikes and 98 oil spills & MLP $75.93$\% overall acc. SVM $79.63$\% overall acc. DT(Bundling) 90.74 overall acc.\\
    
    Oil spills\cite{Mera201472} &  Envisat, 47 images & - & 9 & 155 lookalikes and 80 oil spills & MLP(9:11:2) $96.3$\% lookalike acc. $92.9$\% oil spill acc. \\
    & & & & & DT $92.6$\% lookalike acc. $92.9$\% oil spill acc. \\
    
    Oil spills\cite{Delfrate1996} &  ERS-1 SAR, 59 low-res images & - & 9 & 2471 lookalikes and 42 oil spills & DT $96$\% lookalike acc. $86$\% oil spill acc\\
    
    Oil spills\cite{Topouzelis201268} &  ERS-2 SAR, 24 high-res images & - & 9 & 90 lookalikes and 69 oil spills & DT forest $84.4$\%\\ 
    
    Oil spills\cite{brekke2008classifiers} & ENVISAT, 103 images & masking & - & 12244 lookalikes and 41 oil spills & SVM(C-SVC) $77.4$\% lookalike acc. $82.9$\% oil spill acc.\\

    ECG arrhythmias\cite{Moavenian20103088} & MIT-BIH arrhythmia database & - & 10 & - & accuracy SVM 99\% , MLP 98.22\% \\

    Remote Sensing\cite{Zanaty2012177} & Satimage & feature extraction & 26& - & accuracy for SVM 93.16\% ,and for MLP 96.98\%\\
    
    signature recognition\cite{FriasMartinez2006693} & user signature data &feature extraction& 2 & - & Recognition rate SVM 66.5  , MLP 71.2\\
    wind speed prediction\cite{Mohandes2004939}& daily wind speed data & - &high dimensional feature & - & MSE on testing data for the MLP is 0.0090 while it is 0.0078 for the SVM\\
    
    Hashimoto's disease\cite{Omiotek201340} &  66 Thyroid ultrasound images & normalization & 59 & 54 healthy and 85 sick & MLP(6-8-1): $89.4$\% sick class $61.1$\% healthy class. DT: $89.4$\% sick class $94.4$\% healthy class. \\
    
    
\end{tabular}
}
\caption{An Overview of oil spill and related studies with their results and charateristics}
\label{fig:table}
\end{table*}

These studies reported varying results in accuracy. Studies not related to oil spills have similar performance. When looking at the table\ref{fig:table}, it is hard and inappropriate to directly compare classifiers in terms of accuracy since they all use a different dataset. Accuracies up to 96\% are claimed\cite{Topouzelis200924}, but without even specifying the data used for training and testing, these results are highly questionable. A direct comparison using the same dataset has al so been done\cite{Mera201472}\cite{Xu201414}. Even if classifiers are trained with the same data, the validity of data issue persists. Since real world samples are scarcly available, different results may be found when using the same methodology on a different dataset. All researchers would benefit from a common database for better accuracy estimation of a classifier\cite{Topouzelis200810}. An emphasis exists on the minimization of false negatives for detecting oil spills. In an automatic decision support system where many images are analyzed, an oil spill classified as a lookalike will have higher consequences. Another factor that has an impact on accuracy are the selected features used as input to a classifier. Although in most cases a similar amount of features is used, only some are taken together in different studies. The search for an optimal set of features has been tried using a genetic algorithm\cite{Topouzelis200930}, but this may not exist.
MLP is often used in oil spill detection and increasingly in the more broad remote sensing as well. Their ability to simultaniously handle nonlinear data of a multi-dimensional input space is a large benefit. Furthermore, they do not require an explicit well defined relationship between input and output as they determine the relationship on their own using training set data. This ability to learn is why MLP are considered reliable classifiers and useful for oil spill detection\cite{Delfrate200038}. Surprisingly, decison trees achieve a similar performance compared to the more complex SVM and MLP. This can be atrributed to the fact that decision trees perform better when dealing with catagorical features and that SVM and MLP require a large dataset to achieve its maximum prediction accuracy\cite{kotsiantis2007supervised}. Decision trees are faster to train, but the classifiers are only trained once. When a large dataset is available, MLP and SVM are well suited for detecting oil spills, because they both perform well when a nonlinear relationship exist between the input and output features\cite{kotsiantis2007supervised}.
