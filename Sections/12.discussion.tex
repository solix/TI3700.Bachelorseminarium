\newpage
Here is an overview of all the results of previously mentioned studies.

\begin{table*}[t]
\centering
\begin{tabular}{*{6}{c}}
  Study & data type & Preprocessing & \# Features & formations & Results \\
    \hline
  ex & ex & ex & ex & ex & ex \\  
\end{tabular}
\end{table*}

In section 4, an attempt to compare the three classifiers has been made including similar experiments that are comparable to oil spill detection. These papers used a similar amount of features that where extracted in scarcely available images. \\ Not all papers agreed on a particular classifier having the highest accuracy, but this could have been expected since the results are under discussion and can't really be compared. Each experiment uses a different data set, some even included unverified samples. The chosen classifier largely depends on the feature dimension, type of features and the data set. Even for seemingly similar problems this can lead to the usage of a different classifier. In the case of oil spill detection, much improvement in the research about different classifiers can be done by establishing a common database with labelled dark spots that are verified. Preferably, dark spots in all shapes, sizes and on different geographical points. This will allow researches to train a classifier with the same data set.