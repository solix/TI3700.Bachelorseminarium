
Support Vector Machines(SVMs) is a classifier which outputs an optimal hyperplane or set of hyperplanes in high-dimensional space and classifies new examples(unseen data). In other words, SVMs are supervised learning models that employs learning algorithms to recognize patterns and analyze data, used for classification of unknown data\cite{wiki:SVM}.\\
In figure \ref{fig:SVM} main idea behind SVMs is shown. Consider example dataset described by variables $x_1$ and $x_2$. The operation of SVM algorithm is based on finding the optimal hyperplane between training examples. An optimal hyperplane is the one that gives largest minimum distance to the training examples and It maximizes the margin of training data\cite{opencv_library}.

\begin{figure}[H]
    \includegraphics[width=80mm]{./img/SVM.png}
     \caption{Optimal Separating Hyperplane}
    \label{fig:SVM}
\end{figure}

In recent years, SVMs are widely used in bio-informatics \cite{furey2000support,osuna1997training,guyon2002gene} and other discipline due to its ability to accurately deal with high dimensional data\cite{joachims1998text}. They are popular for couple of theoretical reasons: SVMs are robust to very large number of variables and can learn both simple and complex classification models\cite{cristianini2000}.\\

SVMs is a best known member in the general category of kernel methods\cite{shawe2004kernel}. A kernel method has the ability to generate non-linear decision boundaries by using method designed for linear classifiers; This allows the user to apply a classifier to data that has infinite- dimensional vector space representation such as DNA or protein\cite{ben2010user}.    




