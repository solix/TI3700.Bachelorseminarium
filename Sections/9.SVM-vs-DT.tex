A study closely related to SAR used hydro acoustics to classify different species of fish. Sonar is used to create hydro acoustic images, which results in similar images as SAR \cite{griffiths2003synthetic}. Fifteen features were used. Both SVM and DT were considered good classifiers. SVM required more preprocessing time, but had slightly higher results ($2.7$\% on average). DT was easier to implement and easier to read. This is a general advantage of DT over SVM. A general problem pointed out was that DTs shows greater variance \cite{Robotham2011170}.

SVM and DTs were also compared in a study using LandSat images. They used SVMs, DT and Maximum Likelihood Classifier (MLC) to determine different types of land coverage (e.g. forest, water, wetlands, open grasland) in Uganda. Though the spatial dimension of LandSat (which produces optical/visual images) is lower than the average SAR satellite, there are a lot of similarities. In fact, it has been shown that SAR and LandSat images can be fused together, which is only possible if the images are similar enough \cite{dupas2000sar}. Interestingly enough, DT had a consistently higher accuracy rate ($2.42$\% on average) \cite{Otukei2010S27}.

An extension of DT is the Random Forest. A study \cite{Topouzelis201268} has shown that DT forests can also achieve great results. Data was gathered from the ERS-2, resulting in 24 high resolution images with 69 actual oil spills and 90 lookalikes. They found that within the forest, trees that had 4, 6 or 9 out of the 25 most used features were most successful. Additionally, they established that a forest of 70 trees is optimal. Adding more would decrease accuracy, as the voting process becomes more fuzzier. When comparing this 9 feature decision forest with other classifiers, they found a $84.4$\% accuracy to classify oil spills. The forest that used all 25 features had an accuracy of $79.3$\%. The only method that scored better was a combination of a neural network and genetic algorithms. Their performance was only $0.4$\% higher.

Brekke and Solberg \cite{brekke2008classifiers} did a study on how regularized statistical classifiers do compared to C-SVC when classifying oil spills. C-Support Vector Classification is a type of SVM where a penalty C is applied when instances are being classified incorrectly. This type of SVM is specialized in dealing with imbalanced datasets. Here, the SVM scored only $82.9$\% when classifying oil spills, $4.9$\% lower than its opponent which was a basic statistical classifier. 

Brekke and Solberg also looked at the performance of decision trees to classify oil spills in low resolution SAR. Typically, this makes it harder to correctly classify spills. They found a $86$\% accuracy when identifying oil spills, but $4$\% of the lookalikes were wrongly classified. This again was done on with a very imbalanced dataset, containing 2471 lookalikes and only 42 spills.

So when comparing these classifiers' accuracy, there is no clear indication that one classifier is better than the other. In terms of readability and simplicity, DT does the better job. More experiments should be done with the Random Forest as this might be a viable candidate for classifying oil spills.