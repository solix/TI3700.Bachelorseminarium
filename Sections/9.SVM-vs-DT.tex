A study closely related to SAR used hydro acoustics to classify different species of fish. Sonar is used to create hydro acoustic images, which results in similar images as SAR \cite{griffiths2003synthetic}. Fifteen features were used. Both SVM and DT were considered good classifiers. SVM required more preprocessing time, but had slightly higher results ($2.7$\% on average). DT was easier to implement and easier to read. A general problem pointed out was that DTs shows greater variance \cite{Robotham2011170}.

SVM and DTs were also compared in a study using LandSat images. They used SVMs, DT and Maximum Likelihood Classifier (MLC) to determine different types of land coverage (e.g. forest, water, wetlands, open grasland) in Uganda. Though the spatial dimension of LandSat (which produces optical/visual images) is lower than the average SAR satellite, there are a lot of similarities. In fact, it has been shown that SAR and LandSat images can be fused together, which is only possible if the images are similar enough \cite{dupas2000sar}. DT had a consistently higher accuracy rate ($2.42$\% on average) and a consistently higher kappa statistic with an average difference of $0.035$ \cite{Otukei2010S27}.

An extension of DT is the DT forest. A study \cite{Topouzelis201268} has shown that DT forests can also achieve great results. They found that within the forest, trees that had 4, 6 or 9 out of the 25 most used features were most successful. Additionally, they established that a forest of 70 trees is optimal. Adding more would decrease accuracy, as the voting process becomes more fuzzier. When comparing this 9 feature decision forest with other classifiers, they found a $84.4$\% accuracy to classify oil spills. The forest that used all 25 features had an accuracy of $79.3$\%. The only method that scored better was a combination of a neural network and genetic algorithms. Their performance was only $0.4$\% higher.