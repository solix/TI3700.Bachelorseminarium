It is hard to compare Support Vector Machines and Decision Trees as they both have their advantages and disadvantages. One such difference is that DTs are faster when handling new datasets, compared to SVM. This is because of the complexity of SVMs' arithmetic computations, where DTs only need to follow a logical path in a tree. A more interesting difference is that SVM has higher accuracy in general as was observed in numerous studies \cite{arun2010hybrid}. One such study showed a constant higher accuracy of at least $2.15$ \%.
 
A study more closely related to SAR used hydro acoustics to classify different species of fish. Sonar is used to create hydro acoustic images, which results in similar images as SAR \cite{griffiths2003synthetic}. Fifteen features were used. Both SVM and DT were considered good classifiers. SVM required more preprocessing time, but had slightly higher results ($2.7$\% on average). DT was easier to implement and easier to read. A general problem pointed out was that DTs shows greater variance \cite{Robotham2011170}.

SVM and DTs were also compared in a study using LandSat images. They used SVMs, DT and Maximum Likelihood Classifier (MLC) to determine different types of land coverage (e.g. forest, water, wetlands, open grasland) in Uganda. Though the spatial dimension of LandSat (which produces optical/visual images) is lower than the average SAR satellite, there are a lot of similarities. DT had a consistently higher accuracy rate ($2.42$\% on average) and a consistently higher kappa statistic with an average difference of $0.035$ \cite{Otukei2010S27}.



