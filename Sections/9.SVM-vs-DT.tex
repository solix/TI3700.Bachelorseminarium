A study closely related to SAR uses hydro acoustics to classify different species of fish \cite{Robotham2011170}. Sonar is used to create hydro acoustic images, which results in similar images as SAR \cite{griffiths2003synthetic}. Fifteen features are used. Both SVM and DT are considered good classifiers. SVM requires more preprocessing time, but has slightly higher results \ref{fig:table}. DT is easier to implement and easier to interpret. This is a general advantage of DT over SVM. A general problem pointed out is that DTs shows greater variance accuracy, as slight chances in the data resulted in different splits. Robotham et al. recommend using Random Forest to counter this problem.

SVM and DTs are also compared in a study using LandSat images \cite{Otukei2010S27}. They use SVM, DT and Maximum Likelihood Classifier (MLC) to classify different types of land coverage (e.g. forest, water, wetlands, open grassland) in Uganda. Though the spatial dimension of LandSat (which produces optical/visual images) is lower than the average SAR satellite, there are a lot of similarities. In fact, it has been shown that SAR and LandSat images can be fused together, which is only possible if the images are similar enough \cite{dupas2000sar}. Contrary to the previous study \cite{Robotham2011170}, here DT has consistently higher accuracy \ref{fig:table}.

One study shows that Random Forest can achieve great results in the area of oil spill detection \cite{Topouzelis201268}. Data was gathered from the European Remote Sensing satellite ERS-2, resulting in 24 high resolution images with 69 actual oil spills and 90 lookalikes. Features are rated on importance. Topouzelis and Psyllos argue that a combination of the 4, 6 and 9 most important features \cite{topouzelis2003oil} are most suitable to use to build the trees in the forest. Additionally, they show that a forest of 70 trees is optimal. Adding more does not further decrease the errors in classification. Computation time is taken into account in that decision. When comparing this 9 feature Random Forest with other classifiers, they found a $84.4$\% accuracy to classify oil spills \ref{fig:table}. The forest that used all 25 features had an accuracy of $79.3$\%. The only method that scored better was a combination of a neural network and genetic algorithms. Though this study shows the potential of Random Forest, not all modelling choices are mentioned. Further exhaustive testing with combinations of features is required.

%Brekke and Solberg \cite{brekke2008classifiers} studied how regularized statistical classifiers do compared to a type of SVM when classifying oil spills. Here, SVM scores $82.9$\% when classifying oil spills, $4.9$\% lower than its opponent which is a basic statistical classifier \ref{fig:table}. The dataset was heavily imbalanced. 

Solberg and Solberg \cite{Delfrate2004} look at the performance of decision trees to classify oil spills in low resolution SAR. Typically, this makes it harder to correctly classify spills. They find a $86$\% accuracy when identifying oil spills. Additionally, $4$\% of the lookalikes are wrongly classified, but this error is less important than an oil spill not being detected. This is done on with a very imbalanced dataset, containing 2471 lookalikes and only 42 spills \ref{fig:table}, another example of the imbalanced nature of oil spill data sets.

So when comparing these classifiers' accuracy, there is no clear indication that one classifier is better than the other. Decision trees might seem to have the upper hand but not in a conclusive manner. In terms of readability and simplicity, DT does the better job. More experiments should be done with the Random Forest with bagging as this might be a viable candidate for classifying oil spills.