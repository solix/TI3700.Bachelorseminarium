It is impossible to fully compare Support Vector Machines and Decision Trees. Both have advantages and disadvantages, each serving it's own purpose. One such difference is that DTs are faster when handling new datasets, compared to SVM. This is because of the complexity of SVMs' arithmetic computations, where DTs only need to follow a logical path in a tree. A more interesting difference is that SVM has higher accuracy in general as was observed in numerous studies\cite{arun2010hybrid}.
 
A study more closely related to SAR used hydro acoustics to classify different species of fish. Sonar is used to create hydro acoustic images, which results in similar images as SAR \cite{griffiths2003synthetic}. Fourteen features were used. Both SVM and DT were considered good classifiers. SVM required more preprocessing time, but had slightly higher results. DT was easier to implement and easier to read. However, DT showed greater variance \cite{Robotham2011170}.

SVM and DTs were also compared in a study using LandSat images. They used SVM, DT and Maximum Likelihood Classifier (MLC) to determine different types of land coverage (e.g. forest, water, wetlands, open grasland) in Uganda. Though the spatial dimension of LandSat (which produces optical/visual images) is lower than the average SAR satellite, there are a lot of similarities. Out of eight classes, both SVM and DT were able to distinguish seven classes. DT had a slightly higher accuracy rate and higher kappa statistic \cite{Otukei2010S27}.



